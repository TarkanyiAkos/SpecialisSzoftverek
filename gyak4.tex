\documentclass{article}
\usepackage[utf8]{inputenc}
\usepackage[T1]{fontenc}
\usepackage[magyar]{babel}
\usepackage{hyperref}
\usepackage{amsthm}
\usepackage{amsmath}
\usepackage{mathtools}
\usepackage{amsfonts}
\usepackage{amssymb}
\usepackage{array}
\theoremstyle{remark}
\begin{document}
\newtheorem{tétel}{Tétel}
\newtheorem{Tétel2}{Tétel}[section]
\begin{tétel}[Pitagorasz-tétel]
    Bármely derékszögű háromszög átfogójának négyzete megegyezik a befogók négyzetösszegével. Tehát: ha egy háromszög derékszögű, akkor a leghosszabb oldalára emelt négyzet területe a másik két oldalra emelt négyzetek területének összegével egyenlő.
\end{tétel}
\begin{Tétel2}
    Definiáljunk Tétel (megjelenítendő) néven tételkörnyezetet, és helyezzünk el belőle a dokumentumban 2-t! Az egyiknek adjunk nevet/szerzőt is!
\end{Tétel2}
\leavevmode
\newtheorem{lemma}[tétel]{Lemma}
\begin{lemma}\label{lemmabiz}
    Definiáljunk Lemma néven egy tételkörnyezetet, amit a Tétellel együtt sorszámoz! Helyezzünk el a dokumentumban 1 Lemmát!
\end{lemma}
\theoremstyle{definition}
\newtheorem{definíció}{Definíció}[section]
\begin{proof}
    A bizonyítás szövege.
\end{proof}
\begin{proof}[Lemma bizonyítása]
    A bizonyítás szövege. Hivatkozás a Lemmára: \ref{lemmabiz}.
\end{proof}    
\section{Bevezető}
\newtheorem{1a}{1a}
\begin{1a}
Az \( \frac{1}{{n^2}} \) sorösszege: 
\( \sum_{n=1}^{\infty} \frac{1}{{n^2}} = \frac{{\pi^2}}{6} \)
\end{1a}
\newtheorem{1b}{1b}
\begin{1b}
Az \(n!\) (n faktoriális) a számok szorzata 1től \(n\)ig, azaz
\[n! := \prod_{k=1}^{n} k = 1 \cdot 2 \cdot \ldots \cdot n.\]
Konvenció szerint 0! = 1.
\end{1b}
\newtheorem{1c}{1c}
\begin{1c}
Legyen \(0 \leq k \leq n\). A binomiális együttható
\[
\binom{n}{k} := \frac{n!}{k! \cdot (n - k)!}
\]
ahol a faktoriális (1) szerint értelmezzük.
\end{1c}
\newtheorem{1d}{1d}
\begin{1d}
Az előjel- azaz szignum függvényt a következőképpen definiáljuk:
\[
\text{sgn}(x) = 
\begin{cases} 
1, & \text{ha } x > 0, \\
0, & \text{ha } x = 0, \\
-1, & \text{ha } x < 0.
\end{cases}
\]
\end{1d}
\section{Determináns}
\begin{tétel}{2a}
Legyen
\([n] := \{1, 2, \ldots, n\}\) a természetes számok halmaza 1től \(n\)ig.
\end{tétel}
\begin{tétel}{2b}
Egy \(n\)-edrendű permutáció \(\sigma\) egy bijekció \([n]\) -ből \([n]\) -be. Az \(n\)-edrendű permutációk halmazát, az úgynevezett szimmetrikus csoportot, \(S_n\)nel jelöljük.
\end{tétel}
\begin{tétel}{2c}
Egy \(\sigma \in S_n\) permutációban inverziónak nevezünk egy \((i, j)\) párt, ha \(i < j\) de \(\sigma_i > \sigma_j\).
\end{tétel}
\begin{tétel}{2d}
Egy \(\sigma \in S_n\) permutáció paritásának az inverziók számát nevezzük:
\[I(\sigma) := \sum_{\substack{i, j \in [n] \\ i < j \\ \sigma_i > \sigma_j}} 1.\]
\end{tétel}
\begin{tétel}{2e}
Legyen \(A \in \mathbb{R}^{n \times n}\), egy \(n \times n\)-es (négyzetes) valós mátrix:
\[
A=
\begin{bmatrix}
a_{11} & a_{12} & \ldots & a_{1n} \\
a_{21} & a_{22} & \ldots & a_{2n} \\
\vdots & \vdots & & \vdots \\
a_{n1} & a_{n2} & \ldots & a_{nn} \\
\end{bmatrix}
\]
\end{tétel}
Az \(A\) mátrix determinánsát a következőképpen definiáljuk:
\[
\text{det}(A) := \sum_{\sigma \in S_n} (-1)^{I(\sigma)} \prod_{i=1}^{n} a_{i\sigma(i)}.
\]
\section{Logikai azonosság}
\begin{equation}
(a \land b \land c) \rightarrow d = a \rightarrow (b \rightarrow (c \rightarrow d)) \tag{3}
\end{equation}
Az alábbi azonoságokat használjuk fel bizonyítás nélkül:
\begin{align}
x \rightarrow y &= \overline{x} \lor y \tag{4a} \\
x \lor y &= \overline{x} \land \overline{y} \tag{4b}
\end{align}
A (3) bal oldala, (4) felhasználásával:
\begin{align}
(a \land b \land c) \rightarrow d &= \overline{a} \lor \overline{b} \lor \overline{c} \lor d \tag{5}
\end{align}
A (3) jobb oldala, (4a) ismételt felhasználásával:
\begin{align}
a \rightarrow (b \rightarrow (c \rightarrow d)) &= \overline{a} \lor (\overline{b} \lor (\overline{c} \lor d)) \tag{6}
\end{align}
A feni kifejezések a $\lor$ (diszjunkció) és $\overline{x}$ (nem $x$) műveleteket használják.
\section{Binomiális tétel}
\begin{align}
(a + b)^{n+1} &= (a+b) \cdot \left( \sum_{k=0}^n \binom{n}{k} a^{n-k}b^k \right) \\
&= \sum_{k=0}^n \binom{n}{k} a^{(n+1)-k}b^k \\
&\quad+ \sum_{k=1}^{n+1} \binom{n}{k-1} a^{(n+1)-k}b^{k} \\
&= \binom{n+1}{0} a^{n+1-0} b^0 \\
&\quad+ \sum_{k=1}^n \binom{n+1}{k} a^{(n+1)-k}b^k \\
&\quad+ \binom{n+1}{n+1} a^{n+1-(n+1)} b^{n+1} \\
&= \sum_{k=0}^{n+1} \binom{n+1}{k} a^{(n+1)-k}b^k
\end{align}
\end{document}