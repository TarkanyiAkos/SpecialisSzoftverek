\documentclass[a4paper,12pt,twoside]{book}
% preambulum (preamble):
% csomagok betöltése,
% stílus, saját makrók
\usepackage[magyar]{babel}
\usepackage{t1enc}
\usepackage{xcolor}
\usepackage{graphicx}
\usepackage{lipsum}
\usepackage{hulipsum}
\usepackage{blindtext}
\usepackage{geometry}
\geometry{inner=3cm,outer=5cm,top=3cm,bottom=3cm}
\geometry{bindingoffset=1cm}
\geometry{marginparwidth=3cm,marginparsep=0.5cm}
\begin{document}
%a
\title{Dokumentum Cím}
\author{Szerző}
\date{\today}
\maketitle
\hulipsum[2]
\renewcommand{\thefootnote}{\fnsymbol{footnote}}
\footnote{lábjegyzet}
\setcounter{tocdepth}{3}
\pagenumbering{roman} 
\tableofcontents
\newpage
\pagenumbering{arabic} 
\section[Szek.]{Szekció\footnote{lábjegyzet a címsorba}}
\hulipsum[2-3]
\subsection{Alszekció}
\hulipsum[10]
\marginpar{Ez egy széljegyzet.}
\subsection{Alszekció}
\hulipsum[10]
\marginpar{Ez meg egy másik széljegyzet.}
\section{Szekció}
\hulipsum[2-3]
\subsection{Alszekció}
\hulipsum[1]
\subsubsection{0.2.1.1 Alalszekció}
\hulipsum[1]
\paragraph{0.2.1.1.1 Paragrafus}
\hulipsum[1]
\subparagraph{0.2.1.1.1.1 Alparagrafus}
szia
\appendix
\section{Függelék Szekció}
\hulipsum[1]
\subsection{Függelék alszekció}
\hulipsum[1]
\section{Függelék Szekció}
\hulipsum[1]
\marginpar{Ez is egy széljegyzet.}
\subsection{Függelék alszekció}
\hulipsum[1]

\end{document}